
\documentclass[11pt,a4paper]{article}
\usepackage[utf8]{inputenc}
%\usepackage[icelandic]{babel}
%\usepackage[T1]{fontenc}
\usepackage{amsmath}
\usepackage{amsfonts}
\usepackage{amssymb}
\usepackage{hyperref}

%\documentclass{article}
\usepackage{graphicx}
\graphicspath{ {myndir/} }
\usepackage[T1]{fontenc} 
\usepackage[english]{babel}
\usepackage{fancyhdr}
\usepackage[utf8]{inputenc} 
\usepackage{graphics}
%\usepackage[pdftex]{graphicx}

\usepackage{caption}
\usepackage{subcaption}
\usepackage[top=2in, bottom=1.5in, left=1in, right=1in]{geometry}

\usepackage{listings}
\usepackage{color}
 
\definecolor{codegreen}{rgb}{0,0.6,0}
\definecolor{codegray}{rgb}{0.5,0.5,0.5}
\definecolor{codepurple}{rgb}{0.58,0,0.82}
\definecolor{backcolour}{rgb}{0.95,0.95,0.92}
 
\lstdefinestyle{mystyle}{
    backgroundcolor=\color{backcolour},   
    commentstyle=\color{codegreen},
    keywordstyle=\color{magenta},
    numberstyle=\tiny\color{codegray},
    stringstyle=\color{codepurple},
    basicstyle=\footnotesize,
    breakatwhitespace=false,         
    breaklines=true,                 
    captionpos=b,                    
    keepspaces=true,                 
    numbers=left,                    
    numbersep=5pt,                  
    showspaces=false,                
    showstringspaces=false,
    showtabs=false,                  
    tabsize=2
}
 
\lstset{style=mystyle}

\pagestyle{fancy}
\fancyhf{}
\rhead{Verkfræðileg Forritun - H2015}
\lhead{Háskólinn í Reykjavík}
\chead{DT - 10}
\rfoot{Page \thepage}



\begin{document}


\section*{Verkefni 1: Klasinn DoubleArray}
Þið eigið að búa til klasann DoubleArray og notast við aðskilda þýðingu. Klasinn á að vera ADT, þ.e. breytur og annað er private, fyrir utan þau föll sem eiga að vera sýnileg utan klasans. Klasinn á að innihalda eftirfarandi atriði:
\begin{itemize}
	\item Kvikt fylki sem geymir double tölur
	\item Stærð fylkis
	\item Búa til færibreytulausan smið sem upphafsstillir stærð fylkisins sem 0 og fylkið sjálft sem NULL
	\item Smið sem tekur færibreytur fyrir stærð og gildi. Smiðurinn býr til þá stærð af kviku fylki og setur gildið í hvert einasta stak í fylkinu
	\item Destructor sem skilar öllu minni
	\item Fall sem reiknar meðaltal stakanna í fylkinu
	\[
	 	\bar{x} = \frac{\sum{x}}{n}
	\]
	\item Fall sem reiknar staðalfrávik stakanna í fylkinu
	\[
	 	\sigma = \sqrt{\frac{\sum\left(x - \bar{x}\right)^2}{n - 1}}
	\]
	\item operator$+$ sem leggur saman meðaltöl tveggja DoubleArray hluta
	\item operator$>$ sem ber saman meðaltöl tveggja DoubleArray hluta
	\item operator$<<$ sem prentar út fjölda staka, meðaltal og staðalfrávik
	\item operator$>>$ sem tekur við fjölda staka og les svo inn hvert stak
\end{itemize}
Klasinn á að virka með eftirfarandi main falli (athugið að output þarf ekki endilega að hafa nákvæmlega sama fjölda aukastafa og sýnt er hér):

\begin{lstlisting}[language=C++, caption = Example]
	int main() {
    	DoubleArray DA1, DA2(5, 3.7);
    	cin >> DA1;
    	cout << "DA1: " << DA1 << endl;
    	cout << "DA2: " << DA2 << endl;
    	
    	cin >> DA2;
    	cout << "DA1 + DA2 = " << DA1 + DA2 << endl;
    	
    	if (DA1 > DA2) {
    		cout << "DA1 has the highest average!" << endl;
    	} else {
    		cout << "DA2 has the highest average!" << endl;
    	}
    	
    	return 0;
    }

    Output:
    	Array length: 3
    	4 7 1
    	DA1: n = 3, mean = 4, stddev = 2.45
    	DA2: n = 5, mean = 3.7, stddev = 0
    	1.4 2.7 8.9
    	DA1 + DA2 = 8.3333
    	DA2 has the highest average!
\end{lstlisting}

\section*{Verkfni 2: Tengdur listi}
Nú eigið þið að gera eintengdann lista af heiltölum. Notið structið Node til að geyma gögnin.

\begin{lstlisting}[language=C++, caption = Example]
	struct Node {
		int data;
		Node* next;
	};
\end{lstlisting}

Skrifið forrit sem:
\begin{itemize}
	\item Tekur inn n fjölda af tölum og býr til listann
	\item Prentar út listann
	\item Sækir aftasta stakið í listanum
	\item Athugar hvort tala sem notandi slær inn sé til staðar í listanum
\end{itemize}

Forritið á að hafa eftirfarandi föll:
\begin{itemize}
	\item Fall sem býr til nýtt Node
	\item Fall sem bætir Node við listann
	\item Fall sem prentar út öll stök í listanum
	\item Fall sem athugar hvort einhver tala sé til í listanum
	\item Fall sem finnur síðasta stak í listanum
	\item Fall sem skilar öllu minni
\end{itemize}

\begin{lstlisting}[language=C++, caption=Dæmi um input og output]
	Input:
		8
		1 5 3 4 6 4 2 6
		7
	Output:
		Data in list: 1 5 3 4 6 4 2 6
		Data in back: 1
		The number 7 is not in the list
\end{lstlisting}
\begin{lstlisting}[language=C++, caption=Dæmi um input og output]
	Input:
		4
		1 5 6 1
		5
	Output:
		Data in list: 1 5 6 1
		Data in back: 1
		The number 5 is in the list
		
\end{lstlisting}

\end{document}
